\documentclass[11pt]{article}
\usepackage{graphicx}
\usepackage{amssymb}

\textwidth = 6.5 in
\textheight = 9 in
\oddsidemargin = 0.0 in
\evensidemargin = 0.0 in
\topmargin = 0.0 in
\headheight = 0.0 in
\headsep = 0.0 in
\parskip = 0.2in
\parindent = 0.0in

\newtheorem{theorem}{Theorem}
\newtheorem{corollary}[theorem]{Corollary}
\newtheorem{definition}{Definition}

\begin{document}
\begin{center}
	{Quiz 0}\\
	{CMP 761: Analysis of Algorithms}\\
	{3 September 2002}
\end{center}



Name: \underline{\hspace{ 3 in}}

Student ID (Social Security Number): \underline{\hspace{1in}}

\bigskip
Write your answer to each on a separate piece of paper.  Staple your
answer sheets to this sheet when you turn in the quiz.
\bigskip
\begin{enumerate}
  \item Prove for all natural numbers $n$ that:
	\[
		\sum_{i=1}^{n} i = \frac{n(n+1)}{2}
	\]
\bigskip	
  \item Write (in pseudo-code) an algorithm that sorts a list of $n$
	numbers.
	
	\bigskip
	(More formally, design an algorithm for the following:\\
	{\bf Input:} A sequence of $n$ numbers $\{a_1,a_2,\ldots,a_n\}$.\\
	{\bf Output:} A reordering $\{a'_1,a'_2,\ldots,a'_n\}$
		of the input sequence such that
		$a'_1 \leq a'_2 \leq \cdots \leq a'_n$.)
\end{enumerate}


 \end{document}