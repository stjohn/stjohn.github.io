\documentstyle[12pt]{article}
\pagestyle{empty}
\topmargin= -25pt
\textwidth=6 true in
\textheight=9.5 true in
\oddsidemargin = 0.0 true in
\evensidemargin = 0.0 true in
\newcommand{\ul}{\underline}
\newcommand{\spa}{\hspace{.25in}}
\begin{document}

{\large
\begin{center}
%\mbox{ }
%\vspace{.1in} \\
    Exam II\\
    Computer Programming 420 \\
    Dr.~St.~John\\ 
    Lehman College\\
    City University of New York\\ 
    20 November 2001
\end{center}
}


{\small
{\bf
\begin{center}
Exam Rules
\end{center}
\begin{itemize}
\itemsep=0pt
	\item Show all your work.  Your grade will be based on the work shown.          \item The exam is closed book and closed notes. 
        \item When taking the exam, you may have with you pens or pencils,
		and an 8 1/2" x 11" 
		piece of paper filled with notes, programs, etc. 
	\item You may not use a computer or calculator. 
	\item All books and bags must be left at the front of the classroom 
		during this exam. 
	\item {\bf Do not open the exam until instructed to do so. }
\end{itemize}
}


\begin{enumerate}

% 1 (10pts) T or F
    \item True or False: 
    \begin{enumerate}
        \item \underline{\bf F} Once created, database tables
		and schemas cannot be modified.
        \item \underline{\bf F} You cannot embed a query inside
		another query (ie a subquery) in SQL.
        \item \underline{\bf T} A superkey for a relation is a
		set of attributes that functionally determine all the 
		attributes of the relation.
        \item \underline{\bf T} SQL regards relations as bags
		of tuples, not sets of tuples.
        \item \underline{\bf T} Every set is a bag.
        \item \underline{\bf T} Every functional dependency is a 
		multi-valued dependency.
        \item \underline{\bf T} A view is a definition of how
		one relation (the view) may be constructed from tables 
		stored in the database.
        \item \underline{\bf T} Views can be queried as if
		they were tables.
        \item \underline{\bf F} In SQL, the declarations
		{\tt UNIQUE} and {\tt PRIMARY KEY} have the same effect.
        \item \underline{\bf F} In SQL, the expression,
		{\tt (NULL AND TRUE) OR FALSE} evaluates to {\tt FALSE}.
    \end{enumerate}


%2 (10pts) Simple Relational Algebra queries 
%	2 parts, 5, 5 points
\item Consider the following relational schema:
\begin{verbatim}
    Name(ID, name)   // ID is a key
    GPA(ID, gpa)     // ID is a key
\end{verbatim}
\begin{enumerate}
    \item Write a {\bf relational algebra} expression to find all 
	students and their GPA.  (That is, your answer should be
	a relation, with two attributes, one for the student name
	and one for the GPA).

{\tt
$R := \pi_{name,gpa}(\mbox{Name} \bowtie \mbox{GPA});$

}

    \item Write a {\bf relational algebra} expression to find the 
	names of all students with the highest GPA in the database:

{\tt
From Prof. Widom's Spring 2000 exam:

The idea is to make tuples (Name, gpa, N, G) where $\mbox{gpa} < G$--
that is, the second person in every tuple has the higher GPA (see $R_3$
below).  Then,
find all the names that occur only in the second position-- this means that
no other GPA is higher (see $R_6$):  

$$
\begin{array}{l}
R_1 := \rho_{R(N,G)}(R);\\
R_2 := R \times R_1;\\
R_3 := \sigma_{\mbox{gpa} < G}(R_2);\\
R_4 := \rho_{R(\mbox{name})}(\phi_N(R_3));\\
R_5 := \phi_{\mbox{name}}(R_3);\\
R_6 := R_5 - R_4;\\
\end{array}
$$

This part is hard, and credit was given for solutions that used 
an aggregation operator (e.g. MAX).

}
\end{enumerate}

%3 (10pts) Short answer
\item Write a java program that prints "Hello, world" to the screen:

{\tt From Lab 4:

\begin{verbatim}
public class Hello
{  public static void main(String[] args)
   {  String greeting = "Hello, World!\n";
      System.out.print(greeting);
   }
}
\end{verbatim}

The common mistakes were leaving off the class definition (c-style
program) or forgetting the main method.

}

%4  something straightforward about functional dependencies
%	2 parts, 5 points each
% Ullman's F99 exam
\item Suppose we have a relation $R(A,B,C,D,E)$ with the following
functional dependencies: $AB\rightarrow C$, $CD\rightarrow E$,
$C \rightarrow A$, and $C \rightarrow D$.
\begin{enumerate}
  \item What are all the keys for $R$?

{\tt
From Ullman's Fall 1999 exam:

	$AB$ and $BC$, since the closure of each yields all attributes, 
	and no smaller set does.  Note that any key must include $B$
	since it never occurs on the right hand side of any dependency.
	Looking at the closures of all sets of 2 attributes that include
	$B$, we have $AB^+ = ABCDE$, $BC^+ = ABCDE$, $BD^+ = BD$, 
	$BE^+ = BE$.  You can check that the sets of 3 attributes that
	include $B$ and are superkeys, also include the keys either 
	$AB$ or $BC$.
}

  \item Give an example of a functional dependency that is a 
	BCNF violation for R:
  \item Into what two relations does this violation tell us to
	decompose $R$?

{\tt

There were three possible answers for these parts. Any of them were fine: 

     CD->E: R1(C,D,E), R2(A,B,C,D)

     C->A: R1(A,C), R2(B,C,D,E)

     E->D: R1(D,E), R2(A,B,C,E) 


Note that $AB\rightarrow C$ is not a BCNF violation since 
$AB^+ = ABCDE$.

To decompose the relations, we follow the algorithm from 
\S 3.7.4 Decomposition into BCNF.  The idea is that the attributes 
on the left hand side of the BCNF violation will appear in both
of the new relations.  One relation will also have the attributes
that follow from the dependency.  The other will have the rest of the
attributes.
}

\end{enumerate}

%5  Bags vs. Sets (15 points)
%	3 parts, 4 points for 1st 2, 7 points for last part
\item Suppose $R$ and $S$ are relations.
  \begin{enumerate}
    \item Suppose relations $R$ and $S$ have 1 tuple and 2 tuples,
	respectively.  

	What is the minimum number of $R\cup S$ could have, under the
	bag semantics?

{\tt
The bag will contain all elements of $R$ and $S$ (since duplicates
are not eliminated).  So, the answer is 3.
}

	What is the minimum number of $R\cup S$ could have, under 
	set semantics?

{\tt
Since duplicates are eliminated in sets, we could have that all 
the elements in $R$ are contained in $S$.  
So, the answer is 2.
}

    \item Suppose relations $R$ and $S$ have 2 tuples and 3 tuples,
	respectively.  

	What is the minimum number of $R\cup S$ could have, under the
	bag semantics?

{\tt
The bag will contain all elements of $R$ and $S$ (since duplicates
are not eliminated).  So, the answer is 5.
}

	What is the minimum number of $R\cup S$ could have, under 
	set semantics?

{\tt
Since duplicates are eliminated in sets, we could have that all 
the elements in $R$ are contained in $S$.  
So, the answer is 3.
}

    \item Suppose relations $R$ and $S$ have $n$ tuples and $m$ tuples,

	What is the minimum number of $R\cup S$ could have, under the
	bag semantics?

{\tt
The bag will contain all elements of $R$ and $S$ (since duplicates
are not eliminated).  So, the answer is $|R|+|S| = m + n$.
}

	What is the minimum number of $R\cup S$ could have, under 
	set semantics?

{\tt
Since duplicates are eliminated in sets, we could have that all 
the elements in $R$ are contained in $S$, or vice versa.  
So, the answer is $\max(|S|,|R|) = \max(m,n)$.\\
(In the above examples, $R$ has less elements than $S$, so, if you
use that assumption, the answer is $n$).

}
  \end{enumerate}


%6  Simple SQL queries (20 points)
%	5 parts, 4 points each
% Exercises from Labs 2-6
\item Answer the questions below based on the following schema:
\begin{verbatim}
    companies(co_id, co_name, co_postcode, co_lastchg);
    products(pr_code, pr_desc);
    orders(ord_id, ord_company, ord_product, ord_qty, ord_placed, 
           ord_delivered, ord_paid);
    diary(dy_id, dy_company, dy_timestamp, dy_type, dy_notes);
\end{verbatim}

{\tt All of these are exercises from Lab.}

\begin{enumerate}
    \item Write a query that returns the product codes contained in the
	database:

\begin{verbatim}
SELECT pr_code FROM products;
\end{verbatim}

    \item Write a query that returns the product codes and the average number
	ordered of each per order:

\begin{verbatim}
SELECT pr_code, avg(ord_qty)
FROM products,orders
WHERE pr_code = ord_product
GROUP BY pr_code;
\end{verbatim}

    \item Create a view that contains the name of each company and the 
	total number of orders placed for that company: 


\begin{verbatim}
CREATE VIEW CompanyOrderTotals AS
SELECT co_name, SUM(ord_qty)
FROM products,orders
WHERE co_id = ord_company
GROUP BY co_id;
\end{verbatim}

    \item Create an index on {\tt ord\_company}:


\begin{verbatim}
CREATE INDEX order_co_index ON orders(ord_company);
\end{verbatim}

    \item Write a query that select all orders that were placed in a 
	different month from when the product was delivered. For example, 
	the order is placed on 06-29-2001, and the product is delivered 
	on 07-06-2001. Include in the output, the order ID, the product code, 
	the date the order was placed and the date is was delivered:

\begin{verbatim}
SELECT *
FROM orders
WHERE date_part('month', ord_placed) <> date_part('month', ord_delivered)
\end{verbatim}

\end{enumerate}

% 7  Harder Relational Algebra Question
\item Given two relations $R$ and $S$:
  \begin{enumerate}
    \item Give the definition of the natural join $R \bowtie S$:

{\tt
Let $A_1,A_2,\ldots,A_m$ be the attributes that occur in $R$, but
not $S$.\\
Let $B_1,B_2,\ldots,B_n$ be the attributes that occur in both $R$
and $S$.\\
Let $C_1,C_2,\ldots,C_l$ be the attributes that occur in $S$, but
not $R$.\\

Then, $R \bowtie S$ consists of all tuples $t$
(on attributes
$$A_1,A_2,\ldots,A_m, B_1,B_2,\ldots,B_n, C_1,C_2,\ldots,C_l)$$
that are made by joining tuples from $R$ and $S$ that agree on 
$B_1,B_2,\ldots,B_n$.  Note that the new joined tuples $t$ have
only one copy of the attributes $B_1,B_2,\ldots,B_n$.
}

    \item Give the definition of the theta-join $R \bowtie_C S$:

{\tt
The theta-join $R \bowtie_C S$ is constructed by first taking
the cross product of $R$ and $S$, $R \times S$, and then selecting
the tuples for which condition $C$ is true.

Note that the theta-join $R \bowtie_C S$ has attributes:\\
$$A_1,A_2,\ldots,A_m, R.B_1,R.B_2,\ldots,R.B_n, 
S.B_1,S.B_2,\ldots,S.B_n, C_1,C_2,\ldots,C_l$$
(using the names of the attibutes for Part a)) 
}

    \item What is the difference between $R\bowtie S$ and 
	$R \bowtie_C S$ where the condition $C$ is that $R.A = S.A$ 
	for each attribute $A$ appearing in the schemas of both 
	$R$ and $S$?

{\tt
From the practice problems of the homework.

Each will have the same number of tuples, but $R\bowtie_C S$ will
have two copies of the attributes the relations have in common.
The natural join $R \bowtie S$ will consist of all the joined tuples
of $R$ and $S$, with common attributes collapsed.
$R \bowtie_C S$, with the condition given,
will be the cross product of $R$ and $S$ with all common attributes
identified, but two copies of each common attribute (e.g., $R.A$
and $S.A$).  
}

  \end{enumerate}

% hard SQL question
\item
  \begin{enumerate}
    \item Rewrite the following SQL query {\bf without} using
	the {\tt INTERSECT} or {\tt DIFFERENCE} operators:
      % practice problem from HW, 5.3.5a, p 270
\begin{verbatim}
(SELECT name, address FROM MovieStar WHERE gender = 'F')
        INTERSECT
(SELECT name, address FROM MovieExec WHERE netWorth > 10000000);
\end{verbatim}


{\tt
From the practice problems on the homework (5.3.5a, p 279).

There are many ways to solve this one.  One possible answer is:
}
\begin{verbatim}
    SELECT MovieStar.name, MovieStar.address
    FROM MovieStar, MovieExec
    WHERE MovieStar.name = MovieExec.name AND
        MovieStar.address = MovieExec.address AND
        gender = 'F' AND
        netWorth > 10000000;
\end{verbatim}

    \item Show how to express the relational-algebra query
      $$
        \pi_L(\sigma_C(R_1 \times R_2)))
      $$
      in SQL, where $L$ is a list of attributes and $C$ is an arbritrary
      condition:
      % practice problem from HW, 5.2.4, p 263 

{\tt From the practice problems on the homework (5.2.4, p 263).}
\begin{verbatim}
    SELECT L 
    FROM R_1, R_2
    WHERE C;
\end{verbatim}

  \end{enumerate}

\end{enumerate}
\end{document}


%3 Given R and functional dependencies, give keys and superkeys
\item Consider the relation $R(A,B,C,D,E)$ with the function
dependencies: 
$$
	A\rightarrow B,
	B\rightarrow C,
	C \rightarrow A,
	D \rightarrow E, \mbox{ and }
	E\rightarrow D
$$
\begin{enumerate}
    \item What are the keys of $R$?  
	\vspace{1in}
    \item How many superkeys are there of $R$?  Justify your answer.
	\vspace{1in}
\end{enumerate}


%6 odl --> relational database schema
\item Convert the following ODL description of a schema to a relational
database schema.  Remember that {\tt Course} objects have an ``object
identity,'' and you may invent an attribute representing this OID, 
e.g. {\tt CourseID}.  

\begin{verbatim}
interface Course {
    attribute integer number;
    attribute string room;
    relationship Dept deptOf inverse Dept::coursesOf;
};

interface LabCourse : Course {
    attribute integer computerAlloc;
};

interface Dept (key name) {
    attribute string name;
    attribute string chair;
    relationship Set<Course> coursesOf
         inverse Dept::deptOf;
};
\end{verbatim}




%10 BCNF and 4NF
\item Given the relation schema $R(A,B,C,D)$ with the functional dependencies
$$
\begin{array}{c}
	AB \rightarrow C\\
	BC \rightarrow D\\
	CD \rightarrow A\\
	AD \rightarrow B\\
\end{array}
$$
\begin{enumerate}
    \item Indicate all the Boyce Codd Normal Form violations.  
	Do not forget to consider
	dependencies that are not in the given set, but follow from them.
	However, it is not necessary to give violations that have more than
	one attribute on the right side.
	\vspace{2.5in}
    \item Decompose the relations, as necessary, into a collection of 
	relations that are in Boyce Codd Normal Form.
	\vspace{2in}
\end{enumerate}


\end{enumerate}
\end{document}


