\documentstyle[12pt]{article}
\pagestyle{empty}
\topmargin= -25pt
\textwidth=6 true in
\textheight=9.5 true in
\oddsidemargin = 0.0 true in
\evensidemargin = 0.0 true in
\newcommand{\ul}{\underline}
\newcommand{\spa}{\hspace{.25in}}
\begin{document}

{\large
\begin{center}
%\mbox{ }
%\vspace{.1in} \\
    Exam II\\
    Computer Programming 420 \\
    Dr.~St.~John\\ 
    Lehman College\\
    City University of New York\\ 
    20 November 2001
\end{center}
}

{\large
\begin{center}
\begin{tabular}{ll}
\mbox{ }
\vspace{.1in} \\
NAME (Printed) & \ul{\hspace{3in}}\\ 
NAME (Signed) & \ul{\hspace{3in}} \\
E-mail & \ul{\hspace{3in}}\\
\end{tabular}
\end{center}


{\small
{\bf
\begin{center}
Exam Rules
\end{center}
\begin{itemize}
	\item Show all your work.  Your grade will be based on the work shown.          \item The exam is closed book and closed notes. 
        \item When taking the exam, you may have with you pens or pencils,
		and an 8 1/2" x 11" 
		piece of paper filled with notes, programs, etc. 
	\item You may not use a computer or calculator. 
	\item All books and bags must be left at the front of the classroom 
		during this exam. 
	\item {\bf Do not open the exam until instructed to do so. }
\end{itemize}
}

{\large
\begin{center}
\mbox{ }
\vspace{.25in} \\
\begin{tabular}{|c|c|l|}
\hline \hline
\hspace{.05in} Question 1 \hspace{.05in}&\hspace{.05in}10 points\hspace{.05in} 
	& \hspace{.5in} \mbox{ }  \\ \hline
\hspace{.05in} Question 2 \hspace{.05in}&\hspace{.05in}10 points\hspace{.05in} 
	& \hspace{.5in} \mbox{ }  \\ \hline
\hspace{.05in} Question 3 \hspace{.05in}&\hspace{.05in}10 points\hspace{.05in} 
	& \hspace{.5in} \mbox{ }  \\ \hline
\hspace{.05in} Question 4 \hspace{.05in}&\hspace{.05in}10 points\hspace{.05in} 
	& \hspace{.5in} \mbox{ }  \\ \hline 
\hspace{.05in} Question 5 \hspace{.05in}&\hspace{.05in}10 points\hspace{.05in} 
	& \hspace{.5in} \mbox{ }  \\ \hline 
\hspace{.05in} Question 6 \hspace{.05in}&\hspace{.05in}20 points\hspace{.05in} 
	& \hspace{.5in} \mbox{ }  \\ \hline 
\hspace{.05in} Question 7 \hspace{.05in}&\hspace{.05in}15 points\hspace{.05in} 
	& \hspace{.5in} \mbox{ }  \\ \hline 
\hspace{.05in} Question 8 \hspace{.05in}&\hspace{.05in}15 points\hspace{.05in} 
	& \hspace{.5in} \mbox{ }  \\ \hline 
\hline
\hspace{.05in} TOTAL \hspace{.05in}&\hspace{.05in}100 points\hspace{.05in} 
	& \hspace{.5in} \mbox{ }  \\ \hline \hline
\end{tabular}
\end{center}
}

\newpage



\begin{enumerate}

% 1 (10pts) T or F
    \item True or False: 
    \begin{enumerate}
        \item \underline{\hspace{.25in}} Once created, database tables
		and schemas cannot be modified.
        \item \underline{\hspace{.25in}} You cannot embed a query inside
		another query (ie a subquery) in SQL.
        \item \underline{\hspace{.25in}} A superkey for a relation is a
		set of attributes that functionally determine all the 
		attributes of the relation.
        \item \underline{\hspace{.25in}} SQL regards relations as bags
		of tuples, not sets of tuples.
        \item \underline{\hspace{.25in}} Every set is a bag.
        \item \underline{\hspace{.25in}} Every functional dependency is a 
		multi-valued dependency.
        \item \underline{\hspace{.25in}} A view is a definition of how
		one relation (the view) may be constructed from tables 
		stored in the database.
        \item \underline{\hspace{.25in}} Views can be queried as if
		they were tables.
        \item \underline{\hspace{.25in}} In SQL, the declarations
		{\tt UNIQUE} and {\tt PRIMARY KEY} have the same effect.
        \item \underline{\hspace{.25in}} In SQL, the expression,
		{\tt (NULL AND TRUE) OR FALSE} evaluates to {\tt FALSE}.
    \end{enumerate}


%2 (10pts) Simple Relational Algebra queries 
%	3 parts, 3, 3, 4 points
\item Consider the following relational schema:
\begin{verbatim}
    Name(ID, name)   // ID is a key
    GPA(ID, gpa)     // ID is a key
\end{verbatim}
\begin{enumerate}
    \item Write a {\bf relational algebra} expression to find all 
	students and their GPA.  (That is, your answer should be
	a relation, with two attributes, one for the student name
	and one for the GPA).
	\vspace{.5in}
    \item Write a {\bf relational algebra} expression to find the 
	names of all students with the highest GPA in the database:
	\vspace{1in}
\end{enumerate}

%3 (10pts) Short answer
\item Write a java program that prints "Hello, world" to the screen:
%\vspace{3in}

\newpage

%4  something straightforward about functional dependencies
%	2 parts, 5 points each
% Ullman's F99 exam
\item Suppose we have a relation $R(A,B,C,D,E)$ with the following
functional dependencies: $AB\rightarrow C$, $CD\rightarrow E$,
$C \rightarrow A$, and $C \rightarrow D$.
\begin{enumerate}
  \item What are all the keys for $R$?
     \vspace{.75in}
  \item Give an example of a functional dependency that is a 
	BCNF violation for R:
     \vspace{.4in}
  \item Into what two relations does this violation tell us to
	decompose $R$?
     \vspace{.4in}
\end{enumerate}

%5  Bags vs. Sets (15 points)
%	3 parts, 4 points for 1st 2, 7 points for last part
\item Suppose $R$ and $S$ are relations.
  \begin{enumerate}
    \item Suppose relations $R$ and $S$ have 1 tuple and 2 tuples,
	respectively.  

	What is the minimum number of $R\cup S$ could have, under the
	bag semantics?
	\vspace{.5in}

	What is the minimum number of $R\cup S$ could have, under 
	set semantics?
	\vspace{.5in}

    \item Suppose relations $R$ and $S$ have 2 tuples and 3 tuples,
	respectively.  

	What is the minimum number of $R\cup S$ could have, under the
	bag semantics?
	\vspace{.5in}

	What is the minimum number of $R\cup S$ could have, under 
	set semantics?
	\vspace{.5in}

    \item Suppose relations $R$ and $S$ have $n$ tuples and $m$ tuples,

	What is the minimum number of $R\cup S$ could have, under the
	bag semantics?
	\vspace{1in}

	What is the minimum number of $R\cup S$ could have, under 
	set semantics?
	\vspace{1in}
  \end{enumerate}

\newpage

%6  Simple SQL queries (20 points)
%	5 parts, 4 points each
% Exercises from Labs 2-6
\item Answer the questions below based on the following schema:
\begin{verbatim}
    companies(co_id, co_name, co_postcode, co_lastchg);
    products(pr_code, pr_desc);
    orders(ord_id, ord_company, ord_product, ord_qty, ord_placed, 
           ord_delivered, ord_paid);
    diary(dy_id, dy_company, dy_timestamp, dy_type, dy_notes);
\end{verbatim}
\begin{enumerate}
    \item Write a query that returns the product codes contained in the
	database:
	\vspace{1in}
    \item Write a query that returns the product codes and the average number
	ordered of each per order:
	\vspace{1in}
    \item Create a view that contains the name of each company and the 
	total number of orders placed for that company: 
	\vspace{1in}
    \item Create an index on {\tt ord\_company}:
	\vspace{1in}
    \item Write a query that select all orders that were placed in a 
	different month from when the product was delivered. For example, 
	the order is placed on 06-29-2001, and the product is delivered 
	on 07-06-2001. Include in the output, the order ID, the product code, 
	the date the order was placed and the date is was delivered:
	\vspace{1in}
\end{enumerate}

\newpage
% 7  Harder Relational Algebra Question
\item Given two relations $R$ and $S$:
  \begin{enumerate}
    \item Give the definition of the natural join $R \bowtie S$:
	\vspace{.75in}
    \item Give the definition of the theta-join $R \bowtie_C S$:
	\vspace{.75in}
    \item What is the difference between $R\bowtie S$ and 
	$R \bowtie_C S$ where the condition $C$ is that $R.A = S.A$ 
	for each attribute $A$ appearing in the schemas of 
	both $R$ and $S$?
	\vspace{1in}
  \end{enumerate}

% hard SQL question
\item
  \begin{enumerate}
    \item Rewrite the following SQL query {\bf without} using
	the {\tt INTERSECT} or {\tt DIFFERENCE} operators:
      % practice problem from HW, 5.3.5a, p 270
\begin{verbatim}
    (SELECT name, address FROM MovieStar WHERE gender = 'F')
        INTERSECT
    (SELECT name, address FROM MovieExec WHERE netWorth > 10000000);
\end{verbatim}
	\vspace{1.5in}
    \item Show how to express the relational-algebra query
      $$
        \pi_L(\sigma_C(R_1 \times R_2)))
      $$
      in SQL, where $L$ is a list of attributes and $C$ is an arbritrary
      condition:
      % practice problem from HW, 5.2.4, p 263 
      \vspace{1in}
  \end{enumerate}

\end{enumerate}
\end{document}


%3 Given R and functional dependencies, give keys and superkeys
\item Consider the relation $R(A,B,C,D,E)$ with the function
dependencies: 
$$
	A\rightarrow B,
	B\rightarrow C,
	C \rightarrow A,
	D \rightarrow E, \mbox{ and }
	E\rightarrow D
$$
\begin{enumerate}
    \item What are the keys of $R$?  
	\vspace{1in}
    \item How many superkeys are there of $R$?  Justify your answer.
	\vspace{1in}
\end{enumerate}


\newpage
%6 odl --> relational database schema
\item Convert the following ODL description of a schema to a relational
database schema.  Remember that {\tt Course} objects have an ``object
identity,'' and you may invent an attribute representing this OID, 
e.g. {\tt CourseID}.  

\begin{verbatim}
interface Course {
    attribute integer number;
    attribute string room;
    relationship Dept deptOf inverse Dept::coursesOf;
};

interface LabCourse : Course {
    attribute integer computerAlloc;
};

interface Dept (key name) {
    attribute string name;
    attribute string chair;
    relationship Set<Course> coursesOf
         inverse Dept::deptOf;
};
\end{verbatim}

\newpage


\newpage

%10 BCNF and 4NF
\item Given the relation schema $R(A,B,C,D)$ with the functional dependencies
$$
\begin{array}{c}
	AB \rightarrow C\\
	BC \rightarrow D\\
	CD \rightarrow A\\
	AD \rightarrow B\\
\end{array}
$$
\begin{enumerate}
    \item Indicate all the Boyce Codd Normal Form violations.  
	Do not forget to consider
	dependencies that are not in the given set, but follow from them.
	However, it is not necessary to give violations that have more than
	one attribute on the right side.
	\vspace{2.5in}
    \item Decompose the relations, as necessary, into a collection of 
	relations that are in Boyce Codd Normal Form.
	\vspace{2in}
\end{enumerate}


\end{enumerate}
\end{document}


