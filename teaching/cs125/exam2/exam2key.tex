\documentstyle[12pt]{article}
\pagestyle{empty}
\textwidth=6 true in
\textheight=9 true in
\topmargin = 10pt
\oddsidemargin = 0.0 true in
\evensidemargin = 0.0 true in
\newcommand{\ul}{\underline}
\newcommand{\spa}{\hspace{.25in}}
\begin{document}

{\Large
\begin{center}
\mbox{ }
\vspace{.1in} \\
    Exam 2 \\
    Computer Science 125 \\ 
    Boise State University\\ 
    Friday, 19 March 1999
\end{center}
}

\begin{enumerate}

%1 T/F and short answer
	\item
	\begin{enumerate}
	    \item \underline{T} In Java, functions can never change number parameters.
	    \item \underline{T} Every {\tt for}-loop can be rewritten using a {\tt while}-loop.
	    \item \underline{T} {\tt if}-statements can be nested.
	    \item \underline{T} Loops can be nested.
	    \item \underline{F} A class can have at most one constructor.
	    \item \underline{T} A function can have at most one return value.
	    \item \underline{F} The following is an infinite loop:
\begin{verbatim}    int i = 0; 
    while ( i > 0 )  i--;\end{verbatim}

	    \item \underline{F} The following is an infinite loop:
\begin{verbatim}     for ( int i = 0; i >= 0; i--) 
         System.out.println();\end{verbatim}

	    \item \underline{T} The following code fragment contains a scoping error:
\begin{verbatim}    int j = 0; 
    for ( int i = 0; i >= 0; i--) 
    {  int j = i; }\end{verbatim}	    

	    \item \underline{F} The output of {\tt mystery(1234)}
is {\tt 4321}.
\begin{verbatim}    public void mystery(int n)
    {   if ( n < 10 ) System.out.print(n);
        else {   mystery(n/10); System.out.print(n%10); } }
\end{verbatim}
\end{enumerate}


%2 Loop questions
    \item What is the output of the following:
	\begin{enumerate}
	    \item
\begin{verbatim}
for (int i=5 ; i > 0 ; i--)
{
    Circle c = new Circle(new Point(i,0),i);
    c.draw();
}
\end{verbatim}

INSERT JPEG

	    \item
\begin{verbatim}
for (int i=2 ; i < 6; i= i+2)
{
    Line l = new Line(new Point(-i,i), new Point(i,i));
    l.draw();
}
\end{verbatim}

INSERT JPEG

	    \item
\begin{verbatim}
int i=0; 
while (i < 4)
{
    for (int j=0; j <= i ; j++)
    {
        if ( (i+j)%2 == 0 )
           System.out.print(i+j);
        else
           System.out.print("*");
    } 
    System.out.println();
    i++;
}
\end{verbatim}
        {\bf Output:}
\begin{verbatim}

    0
    *2
    2*4
    *4*6
\end{verbatim}
	\end{enumerate}

%3 object/class questions 8.21, 8.22, 8.23
    \item Consider the following class.  
	\begin{enumerate}
	    \item Mark the constructors
		with a ``C'', the accessor functions with a ``A'',
		and the mutator functions with a ``M'':
\begin{verbatim}

class Gems
{
    public void agate() { n++; }                        M

    public void emerald(String s) { t = s; }            M
    
    public Gems() { n = 0; t = "";}                     C
    
    public int topaz() { return n; }                    A

    public Gems(int a, String s) { n = a; t = s; }      C

    public void diamond() { System.out.println(t); }    A
    
    public String ruby() { return t; }                  A
    
    public Gems(int a) { n = a;t = "";}                 C
    
    private int n;
    private String t;
}

\end{verbatim}

		\item Using the class from above, determine what the following
			program prints:
\begin{verbatim}
public class Mystery
{
  public static void main(String[] arg)
  {
     Gems mine = new Gems();
     Gems yours = new Gems(5, "precious");
     mine.emerald("industrial");
     mine.diamond();
     System.out.println("\n" + yours.topaz());
     yours.emerald(mine.ruby());
     yours.diamond();
  }
}
\end{verbatim}
        {\bf Output:}
\begin{verbatim}
    industrial

    5
    industrial
\end{verbatim}
	
		\item  Write a method {\tt turquois()} for the class above that
		divides {\tt n} by 2 if {\tt t} has length less than 10 and 
		replaces {\tt n} with {\tt 3 * n + 1} if {\tt t} is has 10 or 
		more characters.
	
\begin{verbatim}
    public void turquois()
    {
      if ( t.length() < 10 )
        n = n/2;
      else 
        n = 3*n + 1;
    }
\end{verbatim}

	\end{enumerate}

%4 Ethics
	\item 
		\begin{enumerate}
			\item Explain the differences between a Trojan
				horse, a logic bomb, and a virus.
\begin{verbatim}
A Trojan horse is a program that allows access to an already-penetrated
system-- for example by establishing a new account with superuser privileges.

A logic bomb is a program that is triggered to act upon detecting a certain
sequence of events-- for example a program written by a disgruntled employee
that destroys system files when the employee is removed from the system.

A virus is a a self-replicating program that causes damage and infects
other programs, floppy disks, or hard disks.
\end{verbatim}
			\item Assume you discover that you can view another
				student's java homework, because they have
				set their permissions incorrectly.  

				Would it be ethical to look at the files?
				Should you notify the owner, the instructor,
				the system adminstrator?
				What is the ethical thing to do, with
				respect to your ethical view?  Please state
				the ethical framework and justify your 
				answer.  
		\end{enumerate}

%5 convert letter grade to numeric grade
	\item 
		\begin{enumerate}
			\item Write a function that takes a single letter:
				{\tt A, B, C, D, F}
				as input and returns the corresponding
				numeric grade {\tt 4, 3, 2, 1, 0}.
				The function should return {\tt -1}
				if a longer string is passed to it or
				the input is not that of a valid grade.

				Match the {\tt javadoc} comments below:
\begin{verbatim} 
/**
 * calculates the numeric grade, given the letter grade.  
 * @param grade the letter grade 
 * @return the numeric grade or -1 if the input is invalid 
 */
public int convert(String grade) 
{
  if ( grade.length() != 1 )
    return -1;
  else if ( grade.equals("A") )
    return 4;
  else if ( grade.equals("B") )
    return 3;
  else if ( grade.equals("C") )
    return 2;
  else if ( grade.equals("D") )
    return 1;
  else if ( grade.equals("F") )
    return 0;
  else
    return -1;
}
\end{verbatim}

			\item Assume you have written the function that
				converts grades above.	Use it in a
				{\bf complete} program that asks the
				user for a letter grade, calls the
				function {\tt convert}, and prints out
				the numeric grade.  For full credit,
				you must include comments and use the
				function {\tt convert}.
\begin{verbatim}
// This program converts letter grades to numeric grades.

#import ccj.*

public class GradeConvert
{
  public static void main(String[] arg)
  {
    // Ask user for letter grade:
    System.out.print("Please enter letter grade: ");
    String letterGrade = Console.in.readWord();

    // Print out number grade:
    System.out.println("\nThe number grade is " + convert(letterGrade));
  }
}
\end{verbatim}
		\end{enumerate}

%6 bar chart, broken into pieces
	\item
		\begin{enumerate}
			\item Write a function that takes two points as
			input and draws a box with
				those points as the upper left and lower
				right corners.	Include the {\tt javadoc}
				comments for the function:

\begin{verbatim}
/**
 * Draws a box given the upper left and lower right corner
 * @param ul the upper left corner
 * @param lr the lower right corner
 */
public void drawBox(Point ul, Point lr)
{
  Point ur = new Point(lr.getX(), ul.getY());
  Point ll = new Point(ul.getX(), lr.getY());
  
  new Line(ll,ul).draw();
  new Line(ul,ur).draw();
  new Line(ur,lr).draw();
  new Line(lr,ll).draw();
}
\end{verbatim}
			\item Assume you have written the function that
			draws boxes above.
				Use it to write a {\bf complete} applet
				that asks the user for the number of
				bars in a bar chart, and then prompts
				the user for the height of each box, and
				calls the above function to draw the bar.
\begin{verbatim}
import ccj.*

public class BarChart extends GraphicsApplet
{
  public void run()
  {
    int numBars = readInt("Enter number of bars:");
    setCoord(0,100,numBars,0);
    
    for (int i = 0 ; i < numBars ; i++)
    {
        int height = readInt("Enter height:");
        drawBox(new Point(i, height), new Point(i+1,0));
    }
  }
}
\end{verbatim}

			\item Write a {\tt HTML} file that will display
			your applet:
\begin{verbatim}
<applet code="BarChart.class"
        height=300
        width=300
>
</applet>
\end{verbatim}

		\end{enumerate}

%7 fill in details about the class Rectangle
	\item Below is the class {\tt Rectangle} that works like the
		other graphics classes such as {\tt Circle} and {\tt
		Line}.	A rectangle is constructed from two opposite corner points.
		Fill in the missing definitions below:

\begin{verbatim}import ccj.*; 
public class Rectangle {
  /**
   * Constructs a square of length 2 centered at (0,0).  
   */
  public Rectangle() {

     ul = new Point(-1,1);
     lr = new Point(1,-1);

  }

  /** 
   * Constructs a rectangle given two opposite corners..
   * @param p1 upper left corner
   * @param p2 lower right corner
   */
  public Rectangle(Point p1,Point p2)
  { 

      ul = p1;
      lr = p2;

  }
  
  /** 
   * Draws rectangle on the graphics window.
   */
  public void draw()
  { 

      Point ll = new Point(ul.getX(), lr.getY());
      Point ur = new Point(lr.getX(), ul.getY());

      new Line(ll,ul).draw();
      new Line(ul,ur).draw();
      new Line(ur,lr).draw();
      new Line(lr,ll).draw();

  }


  /**
   * Moves rectangle x units in the horizontal direction and
   * y units in the vertical direction.
   * @param x horizontal change
   * @param y vertical change
   */
  public void move(double x, double y)
  { 

      ul.move(x,y);
      lr.move(x,y);

  }
  
  /* Include variables to hold the data: */

     Point ul;
     Point lr;

}
\end{verbatim}

%8 nested boxes
    \item Given the functions:
\begin{verbatim}
  public void again(Point p1, Point p2, boolean change)
  { if ( distance(p1, p2) < 1 )
      return;
    Point q1 = new Point ( p1.getX(), p2.getY() );
    Point q2 = new Point ( p2.getX(), p1.getY() );
    new Line(p1,q1).draw();
    new Line(q1,p2).draw();
    new Line(p2,q2).draw();
    new Line(q2,p1).draw();
    if ( change )
    { Point m1 = midpoint(p1,q1);
      Point m2 = midpoint(p2,q2);
      new Line(m1,m2).draw();
      again(m1, p2, !change);
    }
    else
    { Point m1 = midpoint(p1,q2);
      Point m2 = midpoint(p2,q1);
      new Line(m1,m2).draw();
      again(p1, m2, !change);
    }
  }

  public double distance(Point a, Point b)
  {
    return ( Math.sqrt( 
        (a.getX() - b.getX())*(a.getX() - b.getX()) +
        (a.getY() - b.getY())*(a.getY() - b.getY())));
  }

  public Point midpoint(Point a, Point b)
  {
    double x = (a.getX() + b.getX())/2;
    double y = (a.getY() + b.getY())/2;
    return new Point(x,y);
  }
\end{verbatim}

 	\begin{enumerate}
	    \item What is the output of 
\begin{verbatim}
   again(new Point(0,0), new Point(1,-1), true)
\end{verbatim}
        {\bf Output:}
        
    INSERT JPEG 
    
	    \item What is the output of 
\begin{verbatim}
    again(new Point(0,2), new Point(4,0), false)
\end{verbatim}
        {\bf Output:}
        
        INSERT JPEG
        
	\end{enumerate}

%9 write a recursive function (include javadoc comments)
%  for part b, write whole program
    \item Write a {\bf recursive} function that takes a {\tt Circle} as
	input and, {\bf if} the radius of the circle is greater than 1, 
	draws two circles inside the given circle (see picture below)
	and calls the function on {\bf each} of these two smaller circles.


     INSERT JPEG
     
\begin{verbatim}
public void nestedCircles(Circle c)
{
  double rad = c.getRadius();
  if ( rad <= 1 )
    return;
    
  Point p = c.getCenter();
  
  // Set up and draw left circle:
  Point lp = new Point(p.getX() - rad/2, p.getY());
  nestedCircles(new Circle(lp, rad/2);
  
  // Set up and draw right circle:
  Point rp = new Point(p.getX() + rad/2, p.getY());
  nestedCircles(new Circle(rp, rad/2);
  
}
\end{verbatim}

%10 design/interface/testing question
	\item You have been hired by {\it Downward Bound Snowboard Shop},
		a store that rents snowboards, to write a program to
		keep track of their rentals.  {\it Downward Bound} 
		stocks several different kinds of snowboards.  For each 
		kind of snowboard, they keep track of the manufacturer of 
		the board, the number of these boards that they 
		own, and the number of these boards that are currently
		rented.  

		When a customer rents or returns a snowboard, the shop 
		updates the number of boards of that kind that are rented.  The
		shop also occasionally retires old snowboards or buys new
		ones, so, the shop needs to be able to update the number
		of boards of each kind it owns. 

		Assume that the shop has three different kinds of snowboards:
		those made by Burton, those made by Ride, and those made
		by Oxygen. Initially, they have 10 of each. 
		\begin{enumerate}
			\item What classes would you use in your design? Why?
\begin{verbatim}
I would use two classes: Snowboad and Shop.  
Snowboard would keep track of the manufacturer, the total number
of board owned, and the number of boards currently rented.
Shop would have 3 member variables for each of the different kind
of snowboards: Burton, Ride, and Oxygen.
\end{verbatim}
			\item For the classes in your design, what methods
				would you have?  Give a {\bf brief} explanation
				of why you chose these methods:
\begin{verbatim}
Snowboard:
  I would have 3 accessor funtions to get the number rented, the
  total stock, and the kind:
  
    getNumRented()
    getTotalStock()
    getKind()
    
  I would also have 3 mutator functions to change the values of
  the member variables:
  
    setNumRented()
    setTotalStock()
    setKind()
    
  There would also be 2 constructors: the default, and one that
  takes a string and 2 integers:
  
    Snowboard(String newKind, int newTotal, int newNumRented)
    
Shop:
   I would have two accessor functions that takes the kind of the
   board.  The first returns the total stock of that kind.  The
   second returns the number rented of that kind:
   
     getTotal()
     getRented()
     
   There would be two mutator functions-- one that updates the
   number of boards rented (given the kind) and the other
   updates the total number of boards:
   
     updateRented()
     updateTotal()
     
   There would also be a default constructor.
\end{verbatim}
			\item What should be included in the test suite for 
				this program?  Why?
\begin{verbatim}
First, there should be simple tests that make sure the information is
entered correctly.
Next, you should write tests to make sure the number of boards owned
or rented is not negative.  
You should also test that you do not rent more boards than 
you own, and that you do not retire (that is, remove from the system)
boards that are currently rented to a customer.
\end{verbatim}
		\end{enumerate}
\end{enumerate}
\end{document}
