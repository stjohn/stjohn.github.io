\documentclass[11pt]{article}
\usepackage{fullpage}

\begin{document}

\title{Homework 7\\
       CSc 72700: Analysis of Algoritms\\
       CUNY Graduate Center, Fall 2001 }
\date{Due Wednesday, 5 December 2001}
\author{}
\maketitle

See the guidelines on the webpage for details about submitting homework.
(If turning your homework in electronically, you can mail it directly
to the grader at: {\tt ivm3@columbia.edu}.)

\section*{Practice Problems}

The problems in this section {\bf are not to be submitted}.  They are to
help you understand the material, and some will appear on exams.

\begin{itemize}
    \item Poly Calls to Poly-time Algorithm:
	36.1-6 on p 924
	(in the second edition, 34.1-5 on p 978).
    \item Listing Vertices in Ham. Cycle:
	36.2-3 on p 928
	(in the second edition, 34.2-3 on p 983).
    \item $P\subseteq co-NP$:
	36.2-9 on p 929
	(in the second edition, 34.2-9 on p 983).
    \item Hamiltonian Path:
	36.5-1 on p 960
	(in the second edition, 34.5-1 on p 1017).
    \item Longest-simple-cycle:
	36.5-6 on p 961
	(in the second edition, 34.5-7 on p 1017).
\end{itemize}

\section*{Graded Problems}

These problems will be graded and should be submitted, following the
guidelines on the webpage.

\begin{enumerate}
    \item Subgraph-isomorphism problem\\
	36.5-1 on p 960
	(in the second edition, 34.5-1 on p 1017).
    \item Bonnie and Clyde\\
	Problem 34-2, p 1018, in second edition:\\
	Bonnie and Clyde have just robbed a bank.  They have a bag of
	money and want to divide it up.  For each of the following 
	scenarios, either give a polynomial-time algorithm, or prove
	that the problem is NP-complete.  The input in each case is
	a list of $n$ items in the bag, along with the value of.
	\begin{enumerate}
	    \item There are $n$ coins, but only 2 different denominations: 
	        some couns are worth $x$ dollars, and some are worth $y$
		dollars.  They wish to divide the money exactly evenly.
	    \item There are $n$ coins with an arbitrary number of 
	    	different denominations, but each denomination is a 
		nonnegative integer power of 2, i.e., the possible 
		denominations are 1 dollar, 2 dollars, 4 dollars, etc.
		They wish to divide the money exactly evenly.
	    \item There are $n$ checks, which are, in an amazing 
	    	coincidence, made out to ``Bonnie or Clyde.''
		They wish to divide the checks so that they each
		get the exact same amount of money.
	    \item There are $n$ checks as in part c), but this time they
	    	are willing to accept a split  in which the difference
		is no greater than 100 dollars.
		\end{enumerate}
	
	
\end{enumerate}

\end{document}
