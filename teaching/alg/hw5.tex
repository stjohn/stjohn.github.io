\documentclass[11pt]{article}
\usepackage{fullpage}

\begin{document}

\title{Homework 5\\
       CSc 72700: Analysis of Algoritms\\
       CUNY Graduate Center, Fall 2001 }
\date{Due Wednesday, 7 November 2001}
\author{}
\maketitle

See the guidelines on the webpage for details about submitting homework.
(If turning your homework in electronically, you can mail it directly
to the grader at: {\tt ivm3@columbia.edu}.)

\section*{Practice Problems}

The problems in this section {\bf are not to be submitted}.  They are to
help you understand the material, and some will appear on exams.

\begin{itemize}
    \item Exercise 19.1-4,p 386 (in the second edition: 18.1-4 on p 441).
    \item Exercise 16.1-3, p 309 
	(in the second edition: Exercise 15.2-4, p 338).
    \item Exercise 17.2-1, p 336 (in the second edition: Exercise 16.2-1).
    \item Exercise 17.3-1, p 344 
	(in the second edition: Exercise 16.3-1, p 392).
    \item Exercise 17.3-2, p 344 
	(in the second edition: Exercise 16.3-2, p 392).
    \item Exercise 17.3-8, p 344 
	(in the second edition: Exercise 16.3-8, p 392).
    \item Consider the two sequences:
\begin{verbatim}
X: AACGTTACCGATATATTT
Y: CAGTACGATGT
\end{verbatim}
	Find:
	\begin{enumerate}
	    \item longest common subsequence (algorithm is in the textbook)
	    \item the pairwise alignment with no penalty for gaps (see 
		the webpage, titled ``Dynamic Programming 
		Tutorial'').
	    \item the pairwise alignment with a penalty for gaps (see 
		the webpage, titled ``Advanced Dynamic 
		Programming Tutorial'')
	\end{enumerate}
\end{itemize}

\section*{Graded Problems}

These problems will be graded and should be submitted, following the
guidelines on the webpage.

\begin{enumerate}
    \item Stacks on Secondary Storage, 
	18-1 on p 452 (in the second edition, 19-1 on p 398).
    \item Printing Neatly, 16-2, p 325
	(in the second edition, 15-2 on p 364).
    \item Making Change, 17-1, p 353
	(in the second edition, 16-1 on p 402).
\end{enumerate}

\end{document}
