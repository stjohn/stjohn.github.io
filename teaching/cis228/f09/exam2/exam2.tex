\documentstyle[12pt]{article}
\pagestyle{empty}
\topmargin= -25pt
\textwidth=6 true in
\textheight=9.5 true in
\oddsidemargin = 0.0 true in
\evensidemargin = 0.0 true in
\newcommand{\ul}{\underline}
\newcommand{\spa}{\hspace{.25in}}
\begin{document}

{\large
\begin{center}
%\mbox{ }
%\vspace{.1in} \\
    Exam II\\
    CIS 228: The Internet \\
    Prof.~St.~John\\ 
    Lehman College\\
    City University of New York\\ 
    3 April 2008
\end{center}
}


{\large
\begin{center}
\begin{tabular}{ll}
\mbox{ }
\vspace{.1in} \\
NAME (Printed) & \ul{\hspace{3in}}\\ 
NAME (Signed) & \ul{\hspace{3in}} \\
E-mail & \ul{\hspace{3in}}\\
\end{tabular}
\end{center}


{\small
{\bf
\begin{center}
Exam Rules
\end{center}
\begin{itemize}
	\item Show all your work.  Your grade will be based on the work shown.          
	\item The exam is closed book and closed notes. 
        \item When taking the exam, you may have with you pens or pencils,
		and an 8 1/2" x 11" 
		piece of paper filled with notes, programs, etc. 
	\item You may not use a computer or calculator. 
	\item All books and bags must be left at the front of the classroom 
		during this exam. 
	\item {\bf Do not open this exams until instructed to do so. }
\end{itemize}
}

{\large
\begin{center}
\mbox{ }
\vspace{.25in} \\
\begin{tabular}{|c|l|}
\hline \hline
\hspace{.05in} Question 1 \hspace{.05in} & \hspace{.5in} \mbox{ }  \\ \hline
\hspace{.05in} Question 2 \hspace{.05in} & \hspace{.5in} \mbox{ } \\ \hline
\hspace{.05in} Question 3 \hspace{.05in} & \hspace{.5in} \mbox{ } \\ \hline
\hspace{.05in} Question 4 \hspace{.05in} & \hspace{.5in} \mbox{ } \\ \hline
\hspace{.05in} Question 5 \hspace{.05in} & \hspace{.5in} \mbox{ }  \\ \hline 
\hspace{.05in} Question 6 \hspace{.05in} & \hspace{.5in} \mbox{ }  \\ \hline
\hspace{.05in} Question 7 \hspace{.05in} & \hspace{.5in} \mbox{ }  \\ \hline
\hspace{.05in} Question 8 \hspace{.05in} & \hspace{.5in} \mbox{ } \\ \hline
\hspace{.05in} Question 9 \hspace{.05in} & \hspace{.5in} \mbox{ } \\ \hline
\hspace{.05in} Question 10\hspace{.05in} & \hspace{.5in} \mbox{ } \\ \hline

\hspace{.05in} TOTAL \hspace{.05in} & \hspace{1in} \mbox{ }  \\ \hline \hline
\end{tabular}
\end{center}
}



\newpage



\begin{enumerate}

%1 T or F
   \item True or False:
    \begin{enumerate}
     	\item \underline{\hspace{.25in}} You can validate CSS files, in the same way you validate HTML files.
     	\item \underline{\hspace{.25in}} Every CSS property declaration ends with a period.
     	\item \underline{\hspace{.25in}} {\tt em} specifies absolute sizes for fonts.
	\item \underline{\hspace{.25in}} If you mix 100\% red, 100\% green, and 100\% blue you will get black.
	\item \underline{\hspace{.25in}} In strict HTML, you can nest block elements inside block elements.	
	\item \underline{\hspace{.25in}} Pseudoclasses must be defined in your HTML file to be used.
	\item \underline{\hspace{.25in}} The {\tt div} element is used to style in-line elements.
	\item \underline{\hspace{.25in}} Tables are defined in columns, using the $<$tc$>$ element.
	\item \underline{\hspace{.25in}} The $<$th$>$ element can designate either row or column headings.  
	\item  \underline{\hspace{.25in}} POST and GET are both ways of sending data to the web application.
    \end{enumerate}

%2 Short answer:
\item Answer in two sentences or less the following:
\begin{enumerate}
    \item What does the padding property do? Give an example. 
    	\vspace{1.2in}
    \item What does the margin property do? How does it differ from the padding property? 	
    	\vspace{1in}
\end{enumerate}     

%3 HTML or CSS     
     \item For each of the following, identify whether it would be done in HTML or CSS:	\begin{enumerate}
	\itemsep 0pt
		\item \underline{\hspace{.5in}}  organize items in a table
		\item  \underline{\hspace{.5in}} include an image
		\item  \underline{\hspace{.5in}} set the width of an element
		\item  \underline{\hspace{.5in}}  change the text alignment of a paragraph
		\item  \underline{\hspace{.5in}}  change the background color of the page
	\end{enumerate}
     

\newpage
%4 Hexcodes     
     \item Fill in the following table with the corresponding decimal or hexcode representation of the 
     	number.  For partial credit, show your work.
	
	{\large
	%\begin{center}
		\begin{tabular}{|c|c|}
			\hline
			{\bf Decimal} & {\bf Hexcode}\\
			\hline
			11 & \\
			\hline
			& \#11\\
			\hline
			20 & \\
			\hline
			& \#ab\\
			\hline
			& \#ff\\
			\hline
		\end{tabular}
	%\end{center}
     	}
	\vspace{1in}

%5 Form question:

    \item Write the HTML that does the following:
    	\begin{enumerate}
		\item makes a text box where the user can enter their name:
			\vspace{.5in}
		\item makes a ``submit'' button with label, ``Order now'':
			\vspace{.5in}
		\item allows the user to select condiments for their order from
			{\tt salt, pepper, garlic} (allow multiple items to be selected):
			\vspace{1.5in}
		\item makes a drop-down menu with a choice of:  ``House Blend,''
			``Shade Grown Bolivia,'' ``Organic Guatemala,'' or ``Kenya.''
	\end{enumerate}

\newpage


%6 Be the browser
     \item Be the browser with the following HTML and CSS files.  Draw the page
     and indicate any style (color, borders, etc) on your page:

{\small     
     \begin{center}

        \begin{minipage}[t]{0.7\linewidth}
         {\bf HTML:}
  {\footnotesize
        \begin{verbatim}
<html xmlns="http://www.w3.org/1999/xhtml" lang="en">
  <head> <title>Favorite Food</title>
    <link type="text/css" rel="stylesheet" 
      href="food.css" />
  </head>
  <body>
    <div id="header"><h1>My Favorite Foods</h1></div>
    <div id="sidebar">
      <p> Recipes for both of these can be found in 
      books on Italian and Spanish cooking at 
      <a href="http://amazon.com">Amazon</a>.</p>
    </div>
    <div id="pesto"> <h2>Pesto</h2>
      <p><a href="pesto.html">Pesto</a> is made from 
      fresh basil, garlic, olive oil, and salt.</p>
    </div>
    <div id="dulce"> <h2>Dulce de Leche</h2>
      <p>The ice cream, <a href="dulce.htm"l>Dulce
        de Leche</a>, is made from milk and 
        sugar and has a smooth caramel taste.</p>
    </div>
    <div id="footer">
       <p>Katherine St. John, 
       <span class="email">stjohn@lehman.cuny.edu</span></p>
    </div>
  </body>
</html>
 
        \end{verbatim}
        }
        \end{minipage}
        \hfill
        \begin{minipage}[t]{0.25\linewidth}
        {\bf CSS:}
\begin{verbatim}
div { 
  border: dotted black 2px;
}
#sidebar {
  width: 100px;
  float: right;
  background-color: blue;
}
#footer {
  clear: both
}
#pesto {
  background-color : green;
}
#dulce {
  background-color: silver;
}
.email {
   font-family: Courier, monospace;
}
\end{verbatim}
        \end{minipage}    	
	
      \end{center} 
}
     
     
\newpage
%7 Simple style sheet    
     \item Write a {\bf complete} CSS style sheet that:
     	\begin{itemize}
		\itemsep 0pt
		\item uses a sans-serif font throughout the page,
		\item has a non-white background color,
		\item $<$h1$>$ and $<$h2$>$ headings have a dotted bottom border,
		\item $<$h1$>$ headings should have white text, and
		\item paragraphs {\bf inside a {\tt div} element} from the class {\tt options} have solid
			borders.

	\end{itemize}
\vspace{2in}
     
 %8 Build a table:
     \item Write the HTML code {\bf and the style sheet} for the table with the following information:\\
		(Hint: Remember to align the text for each cell.)
	\begin{center}{\em Fall 2007: Total Enrollment by Undergraduate and Graduate Level}
	\begin{tabular}{|l|r|r|r|}
		\hline
		&{\bf Undergraduate} &	{\bf Graduate} & 	{\bf Total} \\
		\hline
%		Senior Colleges && 126,651 & 29,445 & 156,096\\
%				& City College & 11,181 & 3,211 & 14,392\\
		 Lehman College & 8,864 & 2,058  &  10,922\\
		\hline
		Community Colleges &76,864 & & 76,864\\ 
		\hline
		{\bf Total University} & 203,515 & 29,445 & 232,960\\
		\hline
	\end{tabular}
	\end{center}
\newpage

 

%9 Layouts     
     \item Write the style sheet that will arrange the following page: 
     	\begin{itemize}
		\itemsep 0pt
		\item the main and sidebar sections should be side-by-side, and not overlapping,
		\item the award should be positioned on top of and overlapping the header, main, 
			and sidebar sections,
		\item the coupon should have a fixed position, halfway down the browser's left edge,
		\item the footer should clear both sections.
	\end{itemize}
     
     	(Hint: Do not change any of the HTML code.)

  {\footnotesize
        \begin{verbatim}
<!DOCTYPE html PUBLIC "-//W3C//DTD XHTML 1.0 Strict//EN"
   "http://www.w3.org/TR/xhtml1/DTD/xhtml1-strict.dtd">
<html xmlns="http://www.w3.org/1999/xhtml" lang="en" xml:lang="en">
  <head> 
   <meta http-equiv="Content-Type" content="text/html; charset=ISO-8859-1" />
    <title>Starbuzz Coffee</title>
    <link rel="stylesheet" type="text/css" href="starbuzz.css" />
  </head> 
  <body>
    <div id="header">
      <img src="images/header.gif" alt="Starbuzz Coffee header image" />
    </div>
    <div id="award">
      <img src="images/award.gif" alt="Roaster of the Year award" />
    </div>
    <div id="main">
      <h1>QUALITY COFFEE, QUALITY CAFFEINE</h1>
      <p>At Starbuzz Coffee, we are dedicated to filling all your  caffeine needs
        through our quality coffees and teas.</p>
      <h1>OUR STORY</h1>
      <p> "A man, a plan, a coffee bean". Okay, that doesn't make a palindrome, but it 
        resulted in a damn good cup of coffee.  Starbuzz's CEO is that man, and you 
        already know his plan: a Starbuzz on every corner.</p> 
      <h1>STARBUZZ COFFEE BEVERAGES</h1>
      <p> We've got a variety of caffeinated beverages to choose
        from at Starbuzz, including our 
        <a href="beverages.html#house" title="House Blend">House Blend</a>,
        <a href="beverages.html#mocha" title="Mocha Cafe Latte">Mocha Cafe Latte</a>, 
        <a href="beverages.html#cappuccino" title="Cappuccino">Cappuccino</a>,
        and a favorite of our customers, 
        <a href="beverages.html#chai" title="Chai Tea">Chai Tea</a>.</p>
      <p>We also offer a variety of coffee beans, whole or ground, for you to
        take home with you.  Order your coffee today using our online
        <a href="form.html" title="form.html">Bean Machine</a>, and take
        the Starbuzz Coffee experience home.</p>
    </div>
    <div id="sidebar">
      <p class="beanheading">
        <img src="images/bag.gif" alt="Bean Machine bag" /> <br />
        ORDER ONLINE with the <a href="form.html">BEAN MACHINE</a> <br />
        <span class="slogan">FAST <br />FRESH <br />TO YOUR DOOR <br /></span>
      </p>
\end{verbatim}
\newpage
\begin{verbatim}
      <p>Why wait?  You can order all our fine coffees right from the Internet 
        with our new, automated Bean Machine.  How does it work?  Just click on 
        the Bean Machine link, enter your order, and behind the scenes, your coffee 
        is roasted, ground (if you want), packaged, and shipped to your door.</p>
    </div>
    <div id="footer">
      &copy; 2005, Starbuzz Coffee<br />
      All trademarks and registered trademarks appearing on 
      this site are the property of their respective owners.
    </div>
    <div id="coupon">
      <a href="freecoffee.html" title="Click here to get your free coffee">
        <img src="images/ticket.gif" alt="Starbuzz coupon ticket" />
      </a>
    </div>
  </body>
</html>

 \end{verbatim}
        }
        
        
        

\newpage
%10 Page and Style sheet from scratch
     \item Write a {\bf complete} HTML file {\bf and} CSS file to display 
     a blog page. The page should have a header (with title) and a footer
     (with your contact information), as well as a navigation sidebar with links to the entries in your page.  
     You should include at least three entries, each containing the date, a short paragraph, 
     and an image.  When the user clicks on a date listed in the sidebar menu, the page should reload
     to that part of the page (hint: use destination anchors).  The sidebar should appear to the right of 
     the blog entries, and the footer should be at the bottom of the page, not overlapping any other 
     section (i.e. it should ``clear'' all other sections). 


\end{enumerate}
\end{document}

\newpage
%10  Strict HTML
\item Modify the following page to be in strict HTML 4.01.  Use the 
appropriate {\tt <doctype>}  and {\tt <meta>} elements and fix non-compliant
parts:
\begin{verbatim}

<html>

  <head>
    <title>Head First Lounge</title>
  </head>

  <body>

    <h1>Welcome to the New and Improved Head First Lounge</h1>

    <p>
      <img src="images/drinks.gif">
    </p>

    <p>
      Join us any evening for refreshing 
      <a href="beverages/elixir.html">elixirs</a>, 
      conversation and maybe a game or two of 
      <em>Dance Dance Revolution</em>.  
      Wireless access is always provided;  
      BYOWS (Bring Your Own Web Server).
    </p>

    <h2>Directions</h2>

    <p>
      You'll find us right in the center 
      of downtown Webville.  If you need help finding
      us, check out our 
      <a href="about/directions.html">detailed directions</a>. 
      Come join us!
    </p>

  </body>
</html>
\end{verbatim}


\end{enumerate}
\end{document}


